\documentclass[abstract=true]{scrartcl}
\subject{Beeldbewerken \\ Assignment 4 : ``Local structure''}
\title{}
\author{Joris Stork, Lucas Swartsenburg}
\usepackage{graphicx,amsmath,subfig}
\addtolength{\parskip}{\baselineskip}
\begin{document}
\maketitle


\section{Analytical local structure}

    \subsection{Calculate derivatives of $f(x,y)=A \sin(V x)+B \cos(W y)$}

        \begin{eqnarray}
            f_{x} = A V \cos(V x) \nonumber \\ 
            f_{y} = -B W \sin(W y) \nonumber \\
            f_{xx} = -A V^2 \sin(V x) \nonumber \\
            f_{yy} = -B W^2 \cos(W y) \nonumber \\
            f_{xy} = 0 \nonumber \\
        \end{eqnarray}

    \subsection{Discretise $f(x,y)=A \sin(V x)+B \cos(W y)$}

        \subsubsection{Code}
        \begin{verbatim}
        
    def f(X,Y, A = 1, B = 2, V = (6 * np.pi / 201), W = (4 * np.pi / 201)):
        """ Discretisation of f. """
        
        F = A * np.sin(V * X) + B * np.cos(W * Y)
        return F

        \end{verbatim}

        Plot is as in assignment description.

    \subsection{Generate images of $Fx$ and $Fy$}

        \subsubsection{Code}
        \begin{verbatim}

    def fx(X,Y, A = 1, B = 2, V = (6 * np.pi / 201), W = (4 * np.pi / 201)):
        """  Discretisation of partial derivative of f wrt x. """
        
        F = A * V * np.cos(V * X)
        return F


    def fy(X,Y, A = 1, B = 2, V = (6 * np.pi / 201), W = (4 * np.pi / 201)):
        """  Discretisation of partial derivative of f wrt y. """

        F = -B * W * np.sin(W * Y)
        return F

        \end{verbatim}

        \subsubsection{Plot}
        \begin{figure}
          \centering
          \includegraphics[width=0.5\textwidth]{../images/0_fx_and_fy}
          \caption{Fx and Fy}
        \end{figure}

    \subsection{Plot gradient vectors}

        \begin{figure}
          \centering
          \includegraphics[width=0.5\textwidth]{../images/1_quiver}
          \caption{Gradient vectors for F}
        \end{figure}


\section{Gaussian convolution}

    \subsection{Implement \texttt{gauss(s)} function}

        \subsubsection{Code}

            \begin{verbatim}
        
    def gauss(s):
        """ Gaussian kernel with scale s and dimensions s*6+1 by s*6+1  """

        size = s * 3
        x, y = np.meshgrid(np.arange(-size,size + 1), np.arange(-size,size + 1))
        kernel = np.exp(-(x**2 / float(s) + y**2 / float(s)))
        kernel = kernel / kernel.sum()
        return x, y, kernel

            \end{verbatim}

    \subsection{Plot of kernel}

        \begin{figure}
          \centering
          \includegraphics[width=0.5\textwidth]{../images/2_kernel3d}
          \caption{Discretised 2D Gaussian kernel}
        \end{figure}

    \subsection{Implement and time the Gaussian convolution}

        \subsubsection{code}

            \begin{verbatim}




            \end{verbatim}

        \subsubsection{Function performance plot}

            The plot in \ref{timing1} shows that the \texttt{gauss()}
            implementation is in an exponential order of complexity. 

            Note that the labels of the y-axis in the following histogram are of
            the form \texttt{convolve(cameraman, gauss([nr])[2])} where
            \texttt{gauss([nr])[2]} represents the Gaussian kernel for scale =
            [nr].
        
            \begin{figure}
              \centering
              \includegraphics[width=0.5\textwidth]{../images/3_time_gauss}
              \caption{Timing of gauss() based convolution against s}
              \label{timing1}
            \end{figure}


\section{Separable Gaussian convolution}

    \subsection{Implement \texttt{gauss1()} function}

    \subsection{Obtain Gaussian convolution}


\section{Gaussian derivatives}

    \subsection{Show that derivatives of 2d Gauss function are separable}

    \subsection{Implement \texttt{gD(F, s, iorder, jorder)} function}

    \subsection{Visualise 2-jet of cameraman image}

        %\begin{figure}
        %  \centering
        %  \subfloat[air1]{\includegraphics[width=0.3\textwidth]{../images/}}                
        %  \caption{blah}
        %\end{figure}


\section{Canny edge detector}

    \subsection{Implement Canny edge detector}

    \subsection{Test Canny function on cameraman image}

        %\begin{figure}
        %  \centering
        %  \subfloat[air1]{\includegraphics[width=0.3\textwidth]{../images/}}                
        %  \caption{blah}
        %\end{figure}

        %\begin{table}
        %    \begin{tabular}{l l | *{10}{c}}
        %
        %              &Model&air2&bre1&air4&leds&air3&fire&star&air1&lbug&bre2\\  
        %        Image &     &    &    &    &    &    &    &    &    &    &    \\ 
        %        \hline
        %        lbug  &     &0.06&0.10&0.02&0.09&0.06&0.07&0.09&0.03&1.00&0.13\\  
        %        bre2  &     &0.27&0.28&0.09&0.15&0.22&0.19&0.16&0.04&0.11&1.00
        %
        %    \end{tabular}
        %    \caption{RGB intersections within database}
        %\end{table}

\end{document}
