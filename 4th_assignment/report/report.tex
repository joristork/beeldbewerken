\documentclass[abstract=true]{scrartcl}
\subject{Beeldbewerken \\ Assignment 4 : ``Local structure''}
\title{}
\author{Joris Stork, Lucas Swartsenburg}
\usepackage{graphicx,amsmath,subfig}
\addtolength{\parskip}{\baselineskip}
\begin{document}
\maketitle


\section{Analytical local structure}

    \subsection{Calculate derivatives of $f(x,y)=A \sin(V x)+B \cos(W y)$}

        \begin{eqnarray}
            f_{x} = A V \cos(V x) \nonumber \\ 
            f_{y} = -B W \sin(W y) \nonumber \\
            f_{xx} = -A V^2 \sin(V x) \nonumber \\
            f_{yy} = -B W^2 \cos(W y) \nonumber \\
            f_{xy} = 0 \nonumber \\
        \end{eqnarray}

    \subsection{Discretise and plot $f(x,y)=A \sin(V x)+B \cos(W y)$}

        %\begin{figure}
        %  \centering
        %  \subfloat[air1]{\includegraphics[width=0.3\textwidth]{../images/}}                
        %  \caption{blah}
        %\end{figure}

    \subsection{Generate images of $Fx$ and $Fy$}

        %\begin{figure}
        %  \centering
        %  \subfloat[air1]{\includegraphics[width=0.3\textwidth]{../images/}}                
        %  \caption{blah}
        %\end{figure}

    \subsection{Plot gradient vectors}

        %\begin{figure}
        %  \centering
        %  \subfloat[air1]{\includegraphics[width=0.3\textwidth]{../images/}}                
        %  \caption{blah}
        %\end{figure}


\section{Gaussian convolution}

    \subsection{Implement \texttt{gauss(s)} function}

    \subsection{Plot kernel}

        %\begin{figure}
        %  \centering
        %  \subfloat[air1]{\includegraphics[width=0.3\textwidth]{../images/}}                
        %  \caption{blah}
        %\end{figure}

    \subsection{Implement and time the Gaussian convolution}

        %\begin{table}
        %    \begin{tabular}{l l | *{10}{c}}
        %
        %              &Model&air2&bre1&air4&leds&air3&fire&star&air1&lbug&bre2\\  
        %        Image &     &    &    &    &    &    &    &    &    &    &    \\ 
        %        \hline
        %        lbug  &     &0.06&0.10&0.02&0.09&0.06&0.07&0.09&0.03&1.00&0.13\\  
        %        bre2  &     &0.27&0.28&0.09&0.15&0.22&0.19&0.16&0.04&0.11&1.00
        %
        %    \end{tabular}
        %    \caption{RGB intersections within database}
        %\end{table}
        
        %\begin{figure}
        %  \centering
        %  \subfloat[air1]{\includegraphics[width=0.3\textwidth]{../images/}}                
        %  \caption{blah}
        %\end{figure}


\section{Separable Gaussian convolution}

    \subsection{Implement \texttt{gauss1()} function}

    \subsection{Obtain Gaussian convolution}


\section{Gaussian derivatives}

    \subsection{Show that derivatives of 2d Gauss function are separable}

    \subsection{Implement \texttt{gD(F, s, iorder, jorder)} function}

    \subsection{Visualise 2-jet of cameraman image}

        %\begin{figure}
        %  \centering
        %  \subfloat[air1]{\includegraphics[width=0.3\textwidth]{../images/}}                
        %  \caption{blah}
        %\end{figure}


\section{Canny edge detector}

    \subsection{Implement Canny edge detector}

    \subsection{Test Canny function on cameraman image}

        %\begin{figure}
        %  \centering
        %  \subfloat[air1]{\includegraphics[width=0.3\textwidth]{../images/}}                
        %  \caption{blah}
        %\end{figure}

        %\begin{table}
        %    \begin{tabular}{l l | *{10}{c}}
        %
        %              &Model&air2&bre1&air4&leds&air3&fire&star&air1&lbug&bre2\\  
        %        Image &     &    &    &    &    &    &    &    &    &    &    \\ 
        %        \hline
        %        lbug  &     &0.06&0.10&0.02&0.09&0.06&0.07&0.09&0.03&1.00&0.13\\  
        %        bre2  &     &0.27&0.28&0.09&0.15&0.22&0.19&0.16&0.04&0.11&1.00
        %
        %    \end{tabular}
        %    \caption{RGB intersections within database}
        %\end{table}

\end{document}
